% !TeX spellcheck = en_US
\documentclass[11pt, fleqn, titlepage]{article}
%\usepackage{siunitx}
\usepackage{texfiles/SpeedyGonzales}
\usepackage{texfiles/MediocreMike}
\newcommand{\so}[2]{{#1}\mathrm{e}{#2}}
% \geometry{top=1cm}
\usepackage{hyperref}
\hypersetup{
	colorlinks=true,
	linkcolor=blue,
	filecolor=magenta,      
	urlcolor=cyan,
}
\title{Phosphate in soil and the effect on barley production}
\author{Oskar Eiler Wiese Christensen s183917 \\ Anders Henriksen s183904 \\ \\ 02445 Project in Statistical Evaluation of Artificial Evaluation}
\date{\today \vspace{2.5cm} \section*{Abstract} \textit{The summary should contain a summary of the problem that  you are working with, which results you got, as well as main conclusions. \\ Don’t get into technical details. The summary should not be very long} \\ Lorem ipsum dolor sit amet, consectetur adipiscing elit, sed do eiusmod tempor incididunt ut labore et dolore magna aliqua. Gravida arcu ac tortor dignissim. Et netus et malesuada fames. Convallis posuere morbi leo urna molestie at elementum eu facilisis. Etiam erat velit scelerisque in dictum non. Mollis nunc sed id semper risus in hendrerit gravida. Cursus euismod quis viverra nibh cras pulvinar mattis nunc sed. Eu tincidunt tortor aliquam nulla. Duis convallis convallis tellus id interdum. Nunc lobortis mattis aliquam faucibus purus in massa tempor. Feugiat sed lectus vestibulum mattis ullamcorper. Malesuada proin libero nunc consequat interdum varius. Sed pulvinar proin gravida hendrerit lectus. Varius morbi enim nunc faucibus a. Ultricies leo integer malesuada nunc vel risus commodo viverra maecenas. Id aliquet lectus proin nibh nisl. Ullamcorper velit sed ullamcorper morbi tincidunt.}

\pagestyle{plain}
\fancyhf{}
\rfoot{Page \thepage{} of \pageref{LastPage}}

\graphicspath{{Billeder/}}

\begin{document}

\maketitle
%\thispagestyle{fancy}
%\tableofcontents

\section{Introduction}
\textit{Briefly introduce the background and setting of the problem, as well as the aim of the report. Furthermore, you could give a very short description of the analysis that will be applied.}

\section{Data}
\textit{Describe of the data you are analyzing. What kinds of data do you have, how were they collected (if applicable)? \\ Include a few good plots to highlight important features in data. You can put additional plots in the appendix.}

\section{Methods}
\textit{Describe the methods you used and why you decided to use them. Also discuss the assumptions behind the methods. Do not go into detail with theory.}

\section{Results}
\textit{Present the results. \\ Tables and figures are good ways of illustrating results.}

\section{Discussion}
\textit{What do your results show? \\ Discuss your results. How reliable are they?}

\section{Conclusion}
\textit{What are your conclusions? The conclusion should be connected to the aim of the report in the introduction. \\ Highlight important results \\ If you have found interesting problems/aspects that you haven’t carried out, you can specify them here as ‘future work’.}

\section{Appendix}




\end{document}
